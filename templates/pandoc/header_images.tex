% Sinew & Steel: Pandoc PDF image/layout helpers
%
% Intended for small B&W spot art that can sit alongside text.
%
% - wrapfig enables text flow around images
% - intextsep/columnsep tighten spacing around wraps (tune as needed)
\usepackage{graphicx}
\usepackage{wrapfig}
\usepackage{enumitem}

% List indentation: keep bullets/numbers *inside* the text block (not hanging
% into the left margin). This also makes top-level lists visually consistent
% with nested lists.
%
% Keep this fairly tight so the quickstart doesn't bloat; we can tune further
% if needed.
\setlist[itemize]{leftmargin=2.5em}
\setlist[enumerate]{leftmargin=2.5em}
\setlength{\intextsep}{0.6\baselineskip}
\setlength{\columnsep}{1em}

% Margin note sizing (used by `.margin-left`/`.margin-right` art).
% Keep this within the page margin so small icons don't get clipped.
\setlength{\marginparwidth}{0.85in}
\setlength{\marginparsep}{0.15in}

% When a `wrapfigure` is active, LaTeX will happily wrap *lists* too. That can
% look awkward for "quick reference" bullets: the label can shift, and the
% list shape can become hard to scan.
%
% `\SSwrapfill` forces the next block to start *below* any active wrapfigure.
% It does this by consuming the remaining wrapped lines as vertical space,
% then finalizing the wrapfig state.
%
% This is a pragmatic, opt-in escape hatch: use it sparingly in markdown
% (e.g. between an image+paragraph and a bullet-heavy section).
\makeatletter
\newcommand{\SSwrapfill}{%
  \par
  \ifnum\c@WF@wrappedlines>\@ne
    \@tempcnta=\c@WF@wrappedlines
    \advance\@tempcnta by -1
    \@tempdima=\baselineskip
    \multiply\@tempdima by \@tempcnta
    \vspace*{\@tempdima}%
    \global\c@WF@wrappedlines\z@
    \WF@finale
  \fi
}
\makeatother
